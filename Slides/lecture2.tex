\documentclass[a4paper,12pt]{article}
\usepackage[margin=1in]{geometry}
\usepackage{setspace}
\usepackage{hyperref}
\usepackage{amsmath}
\usepackage{mdwlist}
\usepackage{listings}
    \lstset{basicstyle=\singlespacing\footnotesize\ttfamily}
\newcommand{\matr}[1]{\mathbf{#1}}
\newcommand{\lik}{\mathcal{L}}

% no indent
\newlength\tindent
\setlength{\tindent}{\parindent}
\setlength{\parindent}{0pt}
\renewcommand{\indent}{\hspace*{\tindent}}

\title{R Programming Basics}
\author{Thomas J. Leeper}
\date{May 20, 2015}

\begin{document}

\maketitle

\doublespacing


\section{Functions}

Built-in functions

\begin{lstlisting}
x <- c(1,3,4,4,5,6,2,3,4)
mean(x)
sum(x)
exp(x)
log(x)
sqrt(x)
length(x)
seq_along(x)
quantile(x, c(0.05, 0.95))
\end{lstlisting}




Writing functions


\begin{lstlisting}
# no arguments
f <- function() 2
f
f()
f <- function() 'hello world'
f()

# one argument
f <- function(x) {
    x^2
}
f(2)
f <- function(x) {
    (1:5)^(x)
}
f(3)

# default argument
f <- function(x = 2) {
    x^2
}
f()
f(2)
f(3)

# multiple arguments
f <- function(x, y) {
    x - y
}

# mutliple arguments, some defaults
f <- function(x, y = 2) {
    x - y
}


# return values
f <- function(x) {
    y <- combn(x, 2)
    min(colSums(y))
}

# list return values
f <- function(x) {
    list(mean = mean(x),
         range = range(x),
         sd = sd(x))
}
f(1:6)
f(1:6)$mean
# etc.

\end{lstlisting}

Some things to remember about functions:

\begin{itemize}

\item Argument recycling. Remember that vectorized operations almost always involve recycling, even if you don't expect it:

\begin{lstlisting}
f <- function(x, y) {
  x + y
}
f(1:6, 2)
f(1:6, 1:2)
\end{lstlisting}

\item Environments: A confusing aspect of functions is the notion of an \textit{evaluation environment}. Inside a function, variables take on values as specified by the function's arguments. If no value is specified for an argument and the argument has no default value, the function will fail to work:

\begin{lstlisting}
f <- function(x) x^2
f()
\end{lstlisting}

\item Variable scope: A more complicated issue relates to how variables that are not function arguments are handled inside functions:

\begin{lstlisting}
z <- 1
f <- function(x) {
  x + z
}
f()
f(1)
\end{lstlisting}

In the above example, \texttt{z} is taken from the function's \textit{parent environment}. This can create confusion because a function can both retrieve variables from its parent environment and modify the values of those variables. Both of these lack \textit{referential transparency} because calling a function on an object can result in the original object being modified. Stata actually works entirely on violating the notion of the referential transparency, so this is an important distinction between how R and Stata behave.\footnote{You can violate referential transparency in R, and there is in principle nothing wrong with that but it can create confusion if you are unprepared for it.}

\item Generic functions and method dispatch: R implements a system of functional programming that involves ``generic'' functions and class-specific methods. This means that the same function (e.g., \texttt{print} or \texttt{summary}) can be used to produce different results for objects of different classes. A classic example of this is \texttt{summary}, which produces radically different output for different kinds of objects:

\begin{lstlisting}
summary(mtcars)
summary(mtcars$cyl)
summary(lm(mpg ~ cyl, data = mtcars))
\end{lstlisting}

\item Lazy evaluation: This is not important for your assignment but it might be something you encounter. Arguments to functions are evaluated \texttt{lazily}, which means that they are not evaluated until they are actually requested by a function, which means they may do things you find unintuitive:

\begin{lstlisting}
x <- 1
z <- 1
f <- function(x, z = x^2) z
f(x = 2) # what do you expect this to be?
f(z = 2) # what do you expect this to be?
\end{lstlisting}

\end{itemize}

Questions?


\section{Repeated Operations}

In order to do many computation tasks, we need to be able to repeat an R expression multiple times or evaluate a function over a range of values. We can write vectorized functions to do some of this, but not always. The following are some strategies for repeating code.


\begin{itemize}

\item Looping: \texttt{for}

\begin{lstlisting}
for(i in c(1,2,3)) {
    print(i)
}

x <- numeric(3)
for(i in seq_along(x)) {
    x <- i * 2
}
\end{lstlisting}



\item Looping: \texttt{while}

\begin{lstlisting}
x <- TRUE
i <- 0
while(x) {
    i <- i + 1
    print(i)
    if(i == 10)
        break
}
\end{lstlisting}


\item The \texttt{apply} and \texttt{*apply}-family functions


\begin{lstlisting}
mean(1:5) # vectorized
lapply(1:5, mean)
sapply(1:5, mean)

m <- matrix(1:16, nrow = 4)
apply(m, 1, sum)
apply(m, 2, sum)


# Use `mapply`
f <- function(x, y) {
  x + y
}
mapply(f, x = 1:3, y = 1:3)
\end{lstlisting}


\item Using \texttt{replicate} to repeat an expression

\begin{lstlisting}
rnorm(1) + rnorm(1)
replicate(5, rnorm(1) + rnorm(1))
\end{lstlisting}


\end{itemize}


Questions?

\section{Simulation Methods}

Simulation methods allow us to solve lots of problems when we either (1) cannot find a closed form mathematical solution to a statistical problem and/or (2) when we think it is inappropriate to rely on a particular parametric assumption (e.g., that a statistic is normally distributed). Simulation methods involve sampling and/or random number generation in tandem with repeated evaluation of a function or expression.

\begin{itemize}

\item Generating random numbers and using \texttt{sample}

\begin{lstlisting}
# pseudo-RNG
rnorm(1)
rnorm(5)
rnorm(5, mean = 3, sd = 2)
rpois(5, 3)

# set RNG seed
set.seed(1)
rnorm(5)
set.seed(1)
rnorm(5)

# sample
sample(2:6, 1)
sample(2:6, 5)
sample(2:6, 10, TRUE)
\end{lstlisting}

\item Simple repeated sampling

\begin{lstlisting}
set.seed(1)
x <- rnorm(1e6, 5, 1)
mean(x)
s1 <- sample(x, 10, FALSE)
mean(s1)

s2 <- sample(x, 1e3, FALSE)
mean(s2)

r1 <- replicate(1000, mean(sample(x, 10, FALSE)))
r2 <- replicate(1000, mean(sample(x, 1e3, FALSE)))
mean(r1)
mean(r2)
sd(r1)
sd(r2)
\end{lstlisting}

\item The Bootstrap. This is useful for estimating variance/SD/SE for estimators that do not have a closed form variance calculation, and in many other contexts (e.g., non-independent observations, etc.). Let's look at a simple example: median.

\begin{lstlisting}
set.seed(1)
x1 <- rnorm(1e6, 5, 1)
mean(x1)
median(x1)

s <- sample(x1, 1e3, FALSE)
mean(s)
median(s)

b <- replicate(1000, mean(sample(s, 1e3, TRUE)))
mean(b) # mean of bootstrapped means
sd(b) # sd of bootstrapped means

quantile(b, c(0.025,0.975)) # 95 % CI for mean


b <- replicate(1000, median(sample(s, 1e3, TRUE)))
mean(b) # mean of bootstrapped medians
sd(b) # sd of bootstrapped medians

quantile(b, c(0.025,0.975)) # 95 % CI for median
\end{lstlisting}



\item Bootstrapping OLS

\begin{lstlisting}
# function to estimate model one time
once <- function(X, y) {
    b <- solve(t(X) %*% X) %*% t(X) %*% y
    fitted <- X %*% b
    e <- fitted - y
    df <- nrow(X) - length(b)
    s2 <- (t(e) %*% e) / df
    V <- as.numeric(s2) * solve(t(X) %*% X)
    sqrt(diag(V))
}

# bootstrap
n <- 5000
bootmat <- matrix(numeric(), ncol = 3, nrow = n)
for(i in 1:n) {
    s <- sample(1:nrow(X), nrow(X), TRUE)
    Xtmp <- X[s,]
    ytmp <- y[s]
    tried <- try(once(Xtmp, ytmp))
    if(!inherits(tried, 'try-error'))
        bootmat[i,] <- tried
}
colMeans(bootmat, na.rm = TRUE)

# jackknife
jackmat <- matrix(numeric(), ncol = 3, nrow = nrow(X))
for(i in 1:nrow(X)) {
    Xtmp <- X[-i,]
    ytmp <- y[-i]
    tried <- try(once(Xtmp, ytmp))
    if(!inherits(tried, 'try-error'))
        jackmat[i,] <- tried
}
colMeans(jackmat, na.rm = TRUE)
\end{lstlisting}



\item Jackknife

\begin{lstlisting}
set.seed(1)
x1 <- rnorm(1e6, 5, 1)
s <- sample(x1, 1e3, FALSE)

mean(s)
j <- numeric(length(s))
for(i in seq_along(s)) j[i] <- mean(s[-i])
mean(s)
sd(s)
quantile(s, c(0.025, 0.975))
\end{lstlisting}

\end{itemize}



Questions? 


\section{Optimization}

Recall the objective of maximum likelihood estimation. We have some data which represent a sample from a population. The population distribution is governed by some parameters, $\theta$, and we want to use our sample data to estimate (or infer) the values of those parameters. MLE is the process of finding the values of $\theta$ that make our sample data most likely. In other words, given $\theta$ takes some set of values, how likely are we to observe the data we actually saw? If $\theta$ had some other set of values, how likely would we be to observe our data? etc. etc.

The Likelihood function is a statement of this conditional probability: $\lik(\theta|data)$. We want to ``maximize'' this function. When the maximum is highest, we have found the input parameter values for $\theta$ that make our data most likely. This in turn tells us that these are the parameters most likely to have generated our sample data, and thus the population data from which the sample data are drawn.

$\lik$ is just a product of the likelihoods of observing each case in our data, so:

\begin{align*}
\lik(\theta|X) & = \lik_1 + \lik_2 + \dots\\
 & = \prod_{i=1}^{n} \lik_i \\
 & = p(x_1|\theta) + p(x_2|\theta) + \dots\\
 & = \prod_{i=1}^{n} p(x_1|\theta)
\end{align*}

It is often more convenient to work with log likelihoods: $\ln\lik = \sum_{i=1}^{n} \ln p(x_i|\theta)$. This transformation is possible because of the property: $log(x*y) = log(x) + log(y)$. Why do we do this? Simplicity but also floating-point issues (because small numbers multiplied by each other go to zero, our likelihood can end up beyond the precision of contemporary computers).

Stating our likelihood function requires making a parametric assumption (i.e., we have to choose a family of probability distributions defined by parameters $\theta$ that will indicating how likely a given observation is).

Once we've defined our likelihood function, we simply evaluate that function (with different possible values for $\theta$) at the observed values of our data and select the values $\theta$ yielding the highest (log-)likelihood.

Likelihood function for a simple example involving a proportion of heads coin tosses. We want to find the parameter $\pi$ that makes this set of heads most likely:
\begin{align*}
\lik(\pi|heads) & = Pr(\text{heads}|\pi) = \pi^\text{heads}(1-\pi)^{\text{tails}}
\end{align*}

Here's the corresponding Log likelihood function:
\begin{align*}
\ln\lik(\pi) & = \text{heads}\ln(\pi) + \text{tails}\ln(1-\pi)
\end{align*}

Here's how we represent that in R (note how $\pi$ is written as \texttt{x}):

\begin{lstlisting}
ll <- function(x, heads = 7, n = 10) {
  heads*log(x) + (n-heads)*log(1-x)
}
\end{lstlisting}

How do we find the maximum likelihood estimate? We can do this in three ways:

\begin{itemize*}
\item Analytical
\item Grid search
\item Optimization
\end{itemize*}

Analytical is only available for some problems. Grid search is slow. Optimization is fastest and most general. 


Analytic solution involves taking the derivative of the log likelihood, which is called the \textbf{score function}:

\begin{align*}
\dfrac{d\ln \lik(\pi)}{d\pi} & = \dfrac{heads}{\pi} + tails \dfrac{1}{1-\pi}(-1)\\
& = \dfrac{\text{heads}}{\pi} - \dfrac{\text{tails}}{1-\pi}
\end{align*}

Solve for parameter $\pi$ to find the maximum likelihood estimate:

\begin{align*}
\dfrac{d\ln \lik(\pi)}{d\pi} & = \dfrac{\text{heads}}{\pi} - \dfrac{\text{tails}}{1-\pi}\\
0 & = \dfrac{\text{heads}}{\pi} - \dfrac{\text{tails}}{1-\pi}\\
\dfrac{\text{tails}}{1-\pi} & = \dfrac{\text{heads}}{\pi}\\
\pi(\text{tails}) & = (1-\pi)\text{heads}\\
\pi(n - x) & = (1-\pi)(x)\\
n\pi - x\pi & = x - x\pi\\
n\pi & = x\\
\pi & = \dfrac{x}{n}
\end{align*}

That provides a simple analytic solution.

To perform a grid search, we simply specify a set of possible values of our parameter, calculate the likelihood for each and pick the one that is largest.

\begin{lstlisting}
possible <- seq(0,1, by = 0.05)
out <- sapply(possible, ll)
out
which.max(out)
possible[which.max(out)]

# plot of log-likelihood
curve(ll, from = 0, to = 1)
points(possible, out, col = 'blue', pch = 18)
points(possible[which.max(out)], out[which.max(out)], col = 'red', pch = 15)
abline(v = possible[which.max(out)], col = 'red')

# override defaults
out <- sapply(possible, ll, heads = 4, n = 10)
\end{lstlisting}


In R, `optim` is a minimization tool, so we always need to negate our function unless we set `control`.

\begin{lstlisting}
# function to minimize
negll <- function(x, heads, n) {
  -(heads*log(x) + (n-heads)*log(1-x))
}

# minimize using `negll`
optim(p = 0.5, negll, heads = 7, n = 10, lower = 0, upper = 1, method = 'Brent')

# maximize using `ll`
optim(p = 0.5, ll, heads = 7, n = 10, lower = 0, upper = 1, 
      method = 'Brent', control = list(fnscale = -1))
\end{lstlisting}


Now let's do a more complex problem, such as the coefficients for a Probit regression model. Let's do something trivial like predict whether a car is automatic as a function of its horsepower:

\begin{lstlisting}
m <- glm(am ~ hp, data = mtcars, family = binomial(link = "probit"))
summary(m)
\end{lstlisting}

How do we obtain the MLE of this model? We have to start by writing out a likelihood function, then maximize that function. The likelihood function will take possible parameter values (i.e., coefficients) as inputs and evaluate our sample data given those possible coefficient values. In other words, if the intercept and slope were actually $\theta$, what is the probability of seeing the values of $\matr{y}$ that we actually observe in our data:

\begin{align*}
\lik(\beta) & = \prod_{i=1}^{n} (\pi_i)^{y_i} (1-\pi_i)^(1-y_i)\\
\ln\lik(\beta) & = \sum_{i=1}^{n} \ln((\pi_i)^{y_i} (1-\pi_i)^{1-y_i}) \\
 & = \sum_{i=1}^{n} y_i \ln(\pi_i) + (1-y_i) \ln(1-\pi_i)\\
 & = \sum_{i=1}^{n} y_i \ln(\phi(X_i\beta)) + (1-y_i) \ln(1-\phi(X_i\beta))
\end{align*}

The last step given $\pi_i = \phi(X_i\beta)$ (the inverse link function for the probit model). For the logit model, this inverse-link function is $pi_i = \dfrac{e^{X_i\beta}}{1+ e^{X_i\beta}}$.

Here's how we represent that in R:

\begin{lstlisting}
ll <- function(beta, X, y) {
  p <- pnorm(X %*% beta)
  lik <- y * log(p) + (1-y) * log(1-p)
  sum(lik)
}
\end{lstlisting}

Now we can solve this using a grid search:

\begin{lstlisting}
X <- cbind(1, mtcars$hp)
y <- mtcars$am

iseq <- seq(-5, 5, by = 0.05)
sseq <- seq(-0.08, 0.08, by = 0.005)
g <- expand.grid(intercept = iseq, slope = sseq)
grid_search <- apply(g, 1, function(z) ll(beta = z, X = X, y = y))

# estimated parameter values
g[which.max(grid_search), ]

# 3-D plot of log-likelihood function
tmp <- grid_search
tmp[tmp == -Inf] <- -1e3
persp(x = iseq, y = sseq, z = matrix(tmp, nrow = length(iseq)), 
      xlim = c(-5, 5), ylim = c(-0.1, 0.1), zlim = c(-1e3, 0), 
      ticktype = 'detailed', theta = 45, phi = 20, 
      shade = 0.25, col = 'lightblue', border = NA)

# heatmap of log-likelihood function
image(x = iseq, y = sseq, z = matrix(tmp, nrow = length(iseq)),
      col = rainbow(1e4, start = 0.1, end = 1))
\end{lstlisting}

Or using \texttt{optim}:

\begin{lstlisting}
opt <- optim(par = c(0,0), ll, X = X, y = y, 
             control = list(fnscale = -1), hessian = TRUE, method = "BFGS")
\end{lstlisting}


The reason we ask for \texttt{hessian = TRUE} is this is what provides us standard errors:

\begin{lstlisting}
opt$hessian
sqrt(diag(solve(-opt$hessian)))

# compare to original results
opt
sqrt(diag(solve(-opt$hessian)))

summary(m)
\end{lstlisting}

Questions?

\end{document}
